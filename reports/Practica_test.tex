\documentclass[a4paper, 11pt, titlepage]{article}
\usepackage[utf8]{inputenc}
\usepackage{kvoptions-patch}
\usepackage[title={Práctica 1: Búsquedas con trayectorias simples}]{estilo}

\makeatletter
 \renewcommand{\ALG@name}{Pseudocódigo}
\makeatother

\pgfplotstableread[col sep=comma]{../results/knn.csv}\dataKNN
\pgfplotstableread[col sep=comma]{../results/SFS.csv}\dataSFS
\pgfplotstableread[col sep=comma]{../results/bestFirst.csv}\dataBF
\pgfplotstableread[col sep=comma]{../results/simulatedAnnealing.csv}\dataSA
\pgfplotstableread[col sep=comma]{../results/tabuSearch.csv}\dataTS
\pgfplotstableread[col sep=comma]{../results/globalMeans.csv}\dataMedias

\begin{document}

    \maketitle

    \pagenumbering{roman}
    \tableofcontents
    \newpage

    \pagenumbering{arabic}

    \section{Testing}

    \subsection{Clasificador $k$-NN}
    \begin{table}[!htb]
        \maketable{\dataKNN}
        \caption{Datos del clasificador $k$-NN}
        \label{knn}
    \end{table}


    \subsection{Algoritmo de comparación}
    \begin{table}[!htb]
        \maketable{\dataSFS}
        \caption{Datos del algoritmo \emph{Sequential forward selection}}
        \label{sfs}
    \end{table}


    \subsection{Búsqueda local primero el mejor}
    \begin{table}[!htb]
        \maketable{\dataBF}
        \caption{Datos de la búsqueda primero el mejor}
        \label{bf}
    \end{table}


    \subsection{Enfriamiento simulado}
    \begin{table}[!htb]
        \maketable{\dataSA}
        \caption{Datos del enfriamiento simulado}
        \label{sa}
    \end{table}


    \subsection{Búsqueda tabú básica}
    \begin{table}[!htb]
        \maketable{\dataTS}
        \caption{Datos de la búsqueda tabú básica}
        \label{ts}
    \end{table}


    \subsection{Datos generales}
    \begin{table}[!htb]
        \maketablemean{\dataMedias}
        \caption{Datos generales}
        \label{medias}
    \end{table}
\end{document}
